\documentclass[conference]{IEEEtran}
\IEEEoverridecommandlockouts
% The preceding line is only needed to identify funding in the first footnote. If that is unneeded, please comment it out.
\usepackage{cite}
\usepackage{amsmath,amssymb,amsfonts}
\usepackage{algorithmic}
\usepackage{graphicx}
\usepackage{textcomp}
\usepackage{xcolor}
\def\BibTeX{{\rm B\kern-.05em{\sc i\kern-.025em b}\kern-.08em
    T\kern-.1667em\lower.7ex\hbox{E}\kern-.125emX}}
\begin{document}

\title{Boosting Proof Number Search via Domain Knowledge Incorporation for the Game of Gobang\\
\thanks{This work was supported by the European Union, co-financed by
  the European Social Fund (EFOP-3.6.3-VEKOP-16-2017-00002), as well
  as by the Hungarian National Excellence Grant
  2018-1.2.1-NKP-00008. It was also supported by the Hungarian Ministry of
  Innovation and Technology NRDI Office within the framework of the
  Artificial Intelligence National Laboratory Program.}
}


\author{\IEEEauthorblockN{1\textsuperscript{st} Domonkos Czifra}
\IEEEauthorblockA{\textit{Alfr\'{e}d R\'{e}nyi Institute of Mathematics} \\
Budapest, Hungary
}
\and
\IEEEauthorblockN{2\textsuperscript{nd} Endre Cs\'{o}ka}
\IEEEauthorblockA{\textit{Alfr\'{e}d R\'{e}nyi Institute of
    Mathematics} \\
Budapest, Hungary
}
\and
\IEEEauthorblockN{3\textsuperscript{rd} Zsolt Zombori}
\IEEEauthorblockA{\textit{Alfr\'{e}d R\'{e}nyi Institute of
    Mathematics} \\
\textit{E{\"o}tv{\"o}s Lor\'{a}nd University}\\
Budapest, Hungary
}
\and
\IEEEauthorblockN{4\textsuperscript{th} G\'{e}za Makay}
\IEEEauthorblockA{\textit{University of Szeged} \\
Szeged, Hungary
}
}

\maketitle

\begin{abstract}
Part of the current introduction text can come here.
\end{abstract}

\begin{IEEEkeywords}
proof number search, gobang, artificial intelligence, game theoretic value
\end{IEEEkeywords}

\section{Introduction}

Our paper presents a case study of incorporating domain knowledge into
the Proof Number Search (PNS) method on the generalized
(m,n,k)-games. PNS and its variants were successfully applied to
solving the game theoretical value of board games (e.g. Gomoku, Hex,
and Go) benefiting from the non-uniform branching factor of the AND-OR
game tree. Enhancing PNS with domain knowledge increases the
non-uniformity of the branching factor, which reduces the search space
and increases the computational gain of PNS compared to alpha-beta
search in many scenarios. 

Our methods can be grouped into three categories. The first category
concerns the reduction of the search space, such as early recognition
of winning states, recognition of mandatory moves,  and partitioning
of the board. The second category is about identifying identical
(isomorphic) states, i.e. turning the search tree into a directed
acyclic graph and sharing information between confluent
branches. Finally, the third category of heuristics guides the
traversal of the search space by overriding the static initialization
rule of the proof and disproof number values with heuristic ones. Our
initialization uses a simple combination of heuristic features and
parameters learned from previously proven states, foreshadowing the
potential of using machine learning to further enhance proof
search. Our paper presents a quantitative evaluation of the effect of
these changes on the search space.

We demonstrate our results on (m,n,k)-games (a generalization of
gomoku and tic-tac-toe), in which two players take turns in placing a
stone of their color on an m�n board (where both m and n can be
infinite), the winner being the player who first gets k stones of
their own color in a row, horizontally, vertically, or
diagonally. There is a weak variant of the game, called the
maker-breaker game where k-in-a-row by the second player (breaker)
does not end the game with a second player win, so the aim of the
breaker is to prevent the maker from winning.

On the theoretical side, we present a tiling technique that can be
used to prove the draw value of the maker-breaker game on an infinite
board. This technique partitions the infinite board into finite pieces
and generalizes the breaker strategy on the small board to the
infinite one.

As a minor result, we can solve the weak (m, n, 4)-game for every m
and n, and we demonstrate that we are very close to prove the
long-standing open conjecture that the weak (inf, inf, 7)-game is a
draw.

\section{Background}
0.75 page

\subsection{$(m,n,k)$--games}

\subsection{The $(inf,inf,7)$--game}

\subsection{Proof Number Search}


\section{Reduction of the $(inf,inf,7)$--game to a finite
  $(4,k,4)$--game}
1 page

Define terms (game theoretic value, solving the game, proof, disproof,
etc.)

State and prove theorem of reduction

State and prove heuristic stop theorem


\section{Proof Number Search for solving the $(4,k,4)$--game}
2 pages

Naive solution (vanilla PNS) is too slow

Methods to reduce the search space:

Early recognition of winning states

Heuristic stop for disproof

Two crossing 2-lines is a proof

Eliminate branches from the search space

Move into 1-line for black

Eliminate squares with no lines on them

When two squares form a 2-line and nothing else crosses them,

eliminate them (pretend white and then black moved on them

When we have a 2 line with one square not having other lines, force

white to move into the other square

Transition table to avoid creating idential search nodes

Merge identical boards

Merge symmetrical boards

Merge isomorphic boards

Partition the boards

Components

Replace initial PN-DN values with heuristics

Naive, hand crafted

Regression based on accumulated data (include image about

winning-losing state positions on the heuristic landscape)

\section{Experiments}
2 pages

\subsection{Experiment 1}

Measure time and maximum tree size reduction for each strategy
separately. Compare against vanilla PNS. Employ 1 hour time
limit. Present the effect of everything together.

All this is 1 baseline, 12 additions, 1 everything together = 14 rows

\subsection{Experiment 2}

This needs to be figured out.

Make a few starting steps so that we are in the disproof
region 

Evaluate the best model from Experiment 1.

Give a reason for thinking that we are approaching a disproof.


\section{Conclusion and Future Work}
0.25 page

Cooperation between finite boards


\section*{Acknowledgment}

No sponsor here, rather people whom we want to thank.

\bibliography{gobang.bib}


\end{document}
